
\documentclass[a4paper,12pt]{article}

\usepackage{ifxetex}
\usepackage{url}

\usepackage{amsfonts}
\usepackage{amsmath}


\ifxetex  % XeLaTeX
   \usepackage{fontspec}
   \defaultfontfeatures{Mapping=tex--text}   % to support TeX conventions like dashes etc
   \usepackage{xltxtra} % extra customisation for XeLaTeX
   \setsansfont{Linux Biolinum O}
   \setmainfont[Ligatures={Common,TeX}, Numbers={OldStyle}]{Linux Libertine O}
\else  % LaTeX
   \usepackage[T1]{fontenc}
   \usepackage[utf8]{inputenc}
   \usepackage[tt=false]{libertine} 
\fi

\usepackage[polish]{babel} 

\usepackage[all]{nowidow}  % dla unikniecia wdow i sierot
\usepackage{indentfirst}

\frenchspacing

 
 
%%%%%%%%%%%%%%%%%%%%%%%%%%%%%%%%%%%%%%%%%
 

\newcommand{\term}[1]{\mathfrak{#1}}
\newcommand{\zbior}[1]{\mathbb{#1}}  % set
\newcommand{\macierz}[1]{{\boldsymbol{\mathrm{#1}}}}  
\newcommand{\wektor}[1]{\macierz{\MakeLowercase{#1}}}
\newcommand{\newItem}[2]{%
  \expandafter\def\csname #1\endcsname {\MakeLowercase{#2}} %
  \expandafter\def\csname n#1\endcsname {\MakeUppercase{#2}} %
  \expandafter\def\csname z#1\endcsname {\zbior{\MakeUppercase{#2}}} %
  \expandafter\def\csname t#1\endcsname {\term{\MakeLowercase{#2}}} %
  \expandafter\def\csname m#1\endcsname {\macierz{\MakeUppercase{#2}}} %
  \expandafter\def\csname w#1\endcsname {\wektor{#2}} %
  }

\newItem{X}{x}
\newItem{Srodek}{v}
\newItem{Rozmycie}{s}
\newItem{Memb}{u}
\newItem{A}{a}
\newItem{C}{c}

%%%%%%%%%%%%%%%%%%%%%%%%%

\title{\sffamily\bfseries Rozmycie grupy wyznaczonej algorytmem FCM}
\author{}
\date{\today}

\begin{document}
\maketitle

Przyjmijmy, że mamy zbiór danych $\zX = \left\{\wX_1, \wX_2, \ldots, \wX_\nX\right\}$, gdzie $\nX$ to liczba danych.
Każda dana $\wX_i$ jest zbudowana z deskryptorów (atrybutów, wymiarów) $\wX_i = \left[ \X_{i1}, \X_{i2}, \ldots, \X_{i\A}, \ldots, \X_{i\nA} \right]$, gdzie $\nA$ to liczba atrybutów. Zapiszmy dane w macierzy $\mX$ w ten sposób, że każdy wiersz to jedna dana $\wX_i$, a każda kolumna do wartości atrybutu $\A$-tego:
\begin{align}
\mX = 
\left[
\begin{matrix}
\wX_1 \\
\wX_2 \\
\vdots \\
\wX_\nX
\end{matrix}
\right] 
=
\left[
\begin{matrix}
\X_{11} & \X_{12} & \ldots & \X_{1\nA} \\
\X_{21} & \X_{22} & \ldots & \X_{2\nA} \\
\vdots  & \vdots  & \ddots & \vdots\\
\X_{\nX 1} & \X_{\nX 2} & \ldots & \X_{\nX \nA} \\
\end{matrix}
\right]
.
\end{align}

Algorytm FCM wypracowuje macierz $\mMemb$ przynależności danych do grup (klastrów). Macierz ma tyle wierszy, ile jest grup $\nC$, i tyle kolumn, ile jest danych: $\mMemb[\nC \times \nX]$:
\begin{align}
\mMemb = 
\left[
\begin{matrix}
\Memb_{11} & \Memb_{12} & \ldots & \Memb_{1\nX} \\
\Memb_{21} & \Memb_{22} & \ldots & \Memb_{2\nX} \\
\vdots  & \vdots  & \ddots & \vdots\\
\Memb_{\nC 1} & \Memb_{\nC 2} & \ldots & \Memb_{\nC \nX} \\
\end{matrix}
\right]
.
\end{align}
Istotne jest, że suma każdej kolumny jest równa 1: 
\begin{align}
\forall_{\C \in \zC} \quad \sum_{\X=1}^{\nX} \Memb_{\C \X} = 1.
\end{align}

Gdy mamy wyznaczone przynależności każdej danej do grup, możemy wyznaczyć środki grup (klastrów). Znowu zastosujemy zapis wektorowy: środek grupy $\C$-tej zapiszemy jako $\wSrodek_\C = \left[\Srodek_{\C 1}, \Srodek_{\C 2}, \ldots, \Srodek_{\C\A}, \ldots, \Srodek_{\C \nA}\right]$. Możemy teraz zapisać środki wszystkich grup jako macierz $\mSrodek$ o $\nC$ wierszach (dla każdej grupy) i $\nA$ kolumnach (dla każdego atrybutu):
\begin{align}
\mSrodek =
\left[
\begin{matrix}
\Srodek_{11} & \Srodek_{12} & \ldots & \Srodek_{1\nA} \\
\Srodek_{21} & \Srodek_{22} & \ldots & \Srodek_{2\nA} \\
\vdots  & \vdots  & \ddots & \vdots\\
\Srodek_{\nC 1} & \Srodek_{\nC 2} & \ldots & \Srodek_{\nC \nA} \\
\end{matrix}
\right]
.
\end{align}
Jak wyznaczyć jednak środki grup? To jest w sumie proste. Środek $\wSrodek_\C$ grupy $\C$-tej wyznaczymy jako średnią ważoną wszystkich danych $\wX_i$ z wartościami przynależności $\Memb_{\C i}$ jako wagami. Zapiszmy to wektorowo:
\begin{align}
\wSrodek_\C = 
\frac
{\sum_{i = 1}^{\nX} \Memb_{\C i}^m \wX_i}
{\sum_{i = 1}^{\nX} \Memb_{\C i}^m}
,
\end{align}
gdzie $m$ jest współczynnikiem, zwykle przyjmujemy, że $m = 2$ (to działa zaskakująco dobrze dla bardzo wielu danych, liczb grup itd).

No to zostało nam wyznaczenie rozmycia grup $\wRozmycie_\C$ grupy $\C$-tej ze względu na każdy atrybut $\wRozmycie_\C = \left[\Rozmycie_{\C 1}, \Rozmycie_{\C 2}, \ldots, \Rozmycie_{\C\A}, \ldots, \Rozmycie_{\C\nA}\right]$. Rozmycie $\Rozmycie_{\C\A}$ grupy $\C$-tej dla atrybutu $\A$-tego wyznacza się jako wariancję:
\begin{align}
\Rozmycie_{\C\A} = 
\sqrt{
\frac
{\sum_{i = 1}^{\nX} \Memb_{\C i}^m \left(\X_{i\A} - \Srodek_{\C\A}\right)^2}
{\sum_{i = 1}^{\nX} \Memb_{\C i}^m}
}
.
\end{align}
\end{document} 
